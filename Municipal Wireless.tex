%!TEX TS-program = xelatex     
%指定源文件处理程序为xelatex
%!TEX encoding = UTF-8 Unicode    
%指定源文件编码为UTF-8
\documentclass[a4paper,14pt,openany]{article}
\usepackage{fontspec,xltxtra,xunicode}  %指定中文包、字体包
\setromanfont{SimSun} %指定字体。凡是系统中有的字体都可以指定

\XeTeXlinebreaklocale “zh”
\XeTeXlinebreakskip = 0pt plus 1pt minus 0.1pt %文章内中文自动换行

\linespread{1.5} %设置行间距

\setlength{\parindent}{2em}  %首行缩进2字符

\usepackage{fancyhdr}% Modify the page headings of book 
\pagestyle{fancy} %页码位置和风格,页眉下面有横线

\newfontfamily{\H}{SimHei} %定义一个宏 \H 是黑体
\newfontfamily{\E}{Arial} %定义另外一个宏\E 英文的时候,用\E指定用Arial字体

\usepackage{draftwatermark}
\SetWatermarkText{Weekly Technology Review by Ct0, Topwalk, \today}%设置水印文字
\SetWatermarkLightness{0.95}%设置水印亮度
\SetWatermarkScale{0.2}%设置水印大小

\title{\H 每周技术评论No.2\\ ---Municipal Wireless}
\author{Jeffery}
\date{\E\today}

\begin{document}

\maketitle %生成封面

\newpage %新的一页
\tableofcontents %生成目录

\newpage
\section{无线城市的概念和发展阶段} %新的章
\paragraph{}
无线城市最初的概念是指用无线网络信号覆盖城市,使人们在城市任何角落都能通过无线的方式接入互联网。
\subsection{作为无线信号覆盖的无线城市}
\paragraph{}
任何正如IT领域内诸多技术概念和产品都是从美国先产生和兴起的一样,无线城市的概念也起源于美国。互联网为人们带来的便利有目共睹,人们希望随时随地能够接入互联网。上世纪最后几年诞生的WiFi技术,使得美国的一些城市雄心勃勃,如费城、旧金山等城市,计划在每个路灯的灯杆上都安装WiFi无线接入点,实现无线信号在城市范围内的覆盖。
\begin{figure}[htbp]
\small
\centering
\includegraphics[width=0.95\textwidth]{SFOMuniWireless.png}
\caption{城市街头的无线接入点}
\label{fig:Municipal Wireless}
\end{figure}

\paragraph{}
这种覆盖城市范围的无线网络称为Municipal Wireless Network。
\paragraph{}
以旧金山为例说明一下的美国的无线城市建设。
\paragraph{}
2006年,旧金山市政府招标进行无线城市建设,最后的中标方是EarthLink和Google,前者负责高带宽(1Mbps)的收费接入点建设,后者负责低带宽(300Kbps)免费接入点建设。到2007年12月,在很多批评和争议之后,EarthLink由于自身财务问题,停止了计划的执行,使得旧金山的无线城市建设陷入了分崩离析的地步。后来网络厂商Cisco和互联网厂商Google/Microsoft/Yahoo!介入,在某些区域提供了免费WiFi信号覆盖。
\paragraph{}
2008年以后,3G的普及、一些商业场所(如星巴克)和公共场所(如图书馆)都开始免费提供WiFi接入,使得全城覆盖WiFi的必要性降低,因此最初规划的无线城市计划大打折扣。费城、波特兰、纽约等城市的无线城市的建设情况各有不同,但面临同样的问题:承建商的商业模式不清晰,广告收入太少,收费接入只能在繁华地段维持盈利,因此无线接入点覆盖范围收窄,而且几乎都是收费的。
\paragraph{}
在技术上,Municipal Wireless Network的建设有几种方式:一种传统城域网建设方式,加入无线接入点;还有采用自组网Mesh技术,使WiFi路由器之间组成自适应网络,路由无需手工配置,网络管理建设和管理都很方便;Google和红杉资本投资的Meraki公司,提供了云方式集中控制管理的WiFi接入和安全设备。
\paragraph{}
在国内,借助奥运会、世博会、亚运会、城运会等,北京、上海和广州等城市大规模建设了无线接入点和室内3G接入点,是国内无线城市建设的肇始。
\paragraph{}
维基百科上列出了全球无线城市的列表 \footnote{http://en.wikipedia.org/wiki/Municipal\_wireless\_network}
\subsection{中移动的无线城市战略}
\paragraph{}
奥运会、世博会中,北京和上海将无线城市的建设作为城市服务设施,由“新型运营商”的网络集成商承接建设。由于政府主导并投资,这些“新型运营商”并未认真考虑进行长期运营、通过运营服务取得盈利的问题,像电信运营商那样。
\paragraph{}
3G、智能终端设备、移动互联网的迅速发展,使电信运营商意识到运营数据业务(而非语音业务)为主的“智能管道”将是自己的未来。以高带宽、低成本的WiFi接入为主体的无线城市将是3G的重要补充,能够弥补3G流量成本较高、带宽较低、终端类型有所限制的缺点。
\paragraph{}
尤其是运营TD-CDMA的中国移动,更加深刻的意识到这一点。中国移动拥有全球最大数量的手机用户,每年利润过千亿(2012上半年利润622亿),远超电信和联通。但是中移动运营的TD网络差、终端类型少,使其自3G时代以来不断损失用户。为了弥补3G的短板,中移动将“无线城市”作为其重点战略,以广州亚运会为契机强势介入无线城市的建设,使原来的无线城市建设主力军,网络集成商、“新兴运营商”们沦为电信运营商的附庸。
\paragraph{}
在中移动的影响之下,中电信、联通也都加入了竞争,三大电信运营商在全国个大城市纷纷开始建设无线接入点。

\newpage
\section{无线城市的应用}
\paragraph{}
最初的无线城市是作为城市服务设施而建设的,提供WiFi免费接入互联网,并不重视应用。国外的承建商需要考虑盈利,因此推出了免费接入加广告、收费接入等商业模式。国内政府主导的无线城市都由政府投资,承建商没有盈利的压力,只提供接入服务,不重视应用。
\subsection{WAP应用时代}
\paragraph{}
电信运营商主导的无线城市建设为了赢得政府部门的支持,开始建设各类应用。通过合作伙伴提供各类无线城市应用方案,中移动等电信运营商向政府部门推销其包括了各类应用在内的无线城市的解决方案,争取赢得政府的支持。运营商的如意算盘是:一旦与政府签订协议,则通过独占性政府资源提供政务信息,然后力图深入行业,吸引集团用户和个人消费者。
\paragraph{}
但无线城市的应用对于电信运营商来说,只是吸引政府支持的诱饵,并非真正关心,因此,无线城市的应用无一例外都很差。广东无线城市是中移动最成功的案例,但其主要应用也只是一些WAP网站,落后、过时、门可罗雀、无人问津、乏善可陈。

\subsection{智慧城市应用}
\paragraph{}
这种情形正在改变。智慧城市和移动互联网的发展,正在影响无线城市的应用。
\paragraph{}
智慧城市主要是物联网概念。通过主动标签(RFID)、语音采集、温度压力传感、位移传感、烟感、视频传感(监控)、电子围栏等各类传感器进行感知并数字化,通过2G/3G/WiFi/Zigbee/有线等网络进行传输,最后在中央端进行展示、控制和管理,提供决策支持。无线城市正好可以作为智慧城市中的传输网络,承载智慧城市的各种应用。政府类的智慧交通、智慧公共安全,行业类的精准农业、物流管理等,都可以由无线城市的网络进行承载。智慧城市的各类物联网应用,正在成为无线城市应用的主要发展方向。电信运营商正在包装、集成各类物联网应用,将其作为无线城市的解决方案,向政府推销,在行业内推广。
\subsection{移动互联网应用}
\paragraph{}
在个人消费者领域,移动互联网的各类应用,如微博、大众点评、微信等要求随时随地在线(所谓“泛在”),客观上对无线城市存在需求。但是,移动互联网的应用一般都来自互联网、面向个人,电信运营商很难将其作为无线城市的应用进行包装和集成,向政府推销。反倒是,移动互联网应用从互联网带来的理念,如Web2.0、社会化、重视用户体验等,可以由无线城市的应用借鉴、学习。这种结合已经看到了一些案例(苏州瑞翼):
\begin{figure}[htbp]
\small
\centering
\includegraphics[width=0.22\textwidth]{WirelessCityAPP1.png}
\includegraphics[width=0.22\textwidth]{WirelessCityAPP2.png}
\includegraphics[width=0.22\textwidth]{WirelessCityAPP3.png}
\includegraphics[width=0.22\textwidth]{WirelessCityAPP4.png}
\caption{苏州瑞翼的无线城市应用}
\label{fig:WirelessCityAPP}
\end{figure}

\paragraph{}
中移动强调所有的移动互联网应用都必须基于其Mobile Market平台,试图借此打造自己的应用中心。但是,iOS和Android用户谁用Mobile Market?运营商对移动互联网的不了解可见一斑。

\newpage
\section{机会}
\subsection{网络安全}
\paragraph{}
电信运营商的无线城市网络建设中,大量的无线接入点需要利用认证、防攻击等网络安全技术。绿盟、启明等公司目前都在为运营商提供这方面的技术服务和产品。
\subsection{传统无线城市应用及平台}
\paragraph{}
电信运营商通过合作伙伴提供无线城市应用和解决方案。可以考虑作为无线城市应用和解决方案提供上,为电信运营商供应无线城市应用平台和各种应用。武汉天喻信息、深圳融创天下等公司是这样的公司。以下是武汉天喻信息2011年年报的分析信息:
\begin{quote}
“无线城市”开发项目成为公司业绩新亮点
\\
公司与移动合作进行“无线城市”项目建设,提供平台建设共15个省,运维和频道应用开发8个省以上。总体份额排名第二(第一、第三分别为华为和融创)。根据不同的需求,每一个省的平台开发收入为100-500万。频道应用包括政务、医疗、旅游、交通、商户信息、便民服务(水、电等的缴费)。公司和湖北移动签署的频道开发应用及运维合同一年的收入为880万元。公司“无线城市”项目主要由子公司天喻通讯提供全套服务,目前有开发人员约400-500人,2011年人员增长较快拖累业绩,2012年计划不再增加人员,将对管理费用控制产生积极意义。
\end{quote}

\subsection{新型应用的机会}
\paragraph{}
结合物联网技术,针对智慧城市,开发面向政府和行业的应用。如交通、城管、消防等物联网应用。如果应用本身对政府或者行业的吸引力足够大,完全可以抛开运营商,直接面对政府或行业,主导整个项目,使运营商成为辅助角色。在某些场合,传输网络已经具备,则项目将与运营商无关,比如武汉智慧公共安全项目。
\paragraph{}另外,电信运营商提供的无线城市应用在移动互联网方面很差,与来自互联网厂商的移动互联网应用有着巨大差距。从意识、应用模式、技术水平、用户体验、运营模式等各个方面都差得很远。
\paragraph{}因此,如果能够将物联网的智慧城市应用与移动互联网的应用模式和用户体验等技术相结合,则有希望推出杀手级应用。
\paragraph{}比如:在山西讨论中提出过的“报警APP”,可以直接将定位信息发送到警方,虽然现在警方有根据电话号码进行基站定位的技术,但是智能手机的GPS定位更为精准,通过APP可以在报警时提供GPS信息,这是传统报警方式不能提供的。当然,这只是一个点,可能不足以达到“杀手”的量级。
\paragraph{}再比如:现在政务大厅中的各项业务,搬到移动互联网上来,在智能手机上提供APP,充分利用智能手机的Mic、摄像头、GPS、触控屏幕等,结合Web2.0(大家帮助大家)、社会化等机制,更加快捷方面的办理各种审批手续。
\newpage
\section{关于CTO评论}
\paragraph{}
《CTO Technology Review》是本人根据业界动态、技术发展状况、公司业务发展方向进行选题,通过大量预研性阅读,对某个技术主题进行的讨论性,主要设计该项技术的简介和发展、技术要点、商业模式、我们的机会等,篇幅在3000至5000字。计划每周一期,每期对一个技术主题进行评论。
\paragraph{}
预定的读者对象是公司高管。大家可对评论的内容进行反馈,包括详细程度和深度、延伸的问题、纠正或补充、深入或进一步了解的需求等。大家的反馈一般在下一期进行补充评论。
\paragraph{}
如果大家一致认为该项技术值得投入,将由我和研发中心组织可行性研究或者原型开发。
\subsection{下一期选题}
\begin{itemize}
\item
IPv6时代的网络安全
\item
NGFW 
\end{itemize}
\subsection{改进意见征集}
\begin{itemize}
\item
哪些地方需要改进?
\item
有什么意见或建议?
\end{itemize}

\end{document}